\problem{}
با روش تقسیم و حل این سوال را حل خواهم کرد و ایده اصلی سوال تقسیم کردن حول
عوض یکتا در هر مرحله است.\\
ابتدا در نظر میگیریم که در هر آرایه دلخواه اگر بخواهیم تمام زیرآرایه ها دارای
عضو یکتا باشند خود آرایه اصلی که جز زیرآرایه هاست نیز دارای عوض یکتاست.
پس در هر آرایه واجد شرایط حداقل یک عضو یکتا وجود دارد و ما تقسیم و حل را
حول آن عضو انجام میدهیم.\\
اگر تمام زیر آرایه های آرایه ی سمت راست عضو یکتا دارای عضو یکتا باشند 
و همچنین تمام زیر آرایه های آرایه سمت چپ عضو یکتا نیز دارای عضو یکتا باشند
به وضوح تمام زیر آرایه هایی که یک قسمت از آنها در زیر آرایه سمت چپ
و قسمت دیگر آنها در زیر آرایه سمت راست هستند به دلیل اینکه شامل عضو یکتا میشوند
حتما دارای عضو یکتا هستند.(که همان عضو یکتایی است که حول آن تقسیم و حل انجام دادیم)
پس کافیست عضو یکتا را بیابیم که با مرتب سازی آرایه ($O(n\ln{n})$)
و پیمایش روی آن ($O(n)$)
میتوان عضو یکتا را پیدا کرد.
پس رابطه بازگشتی ما به صورت زیر است.\\
\[ T(n) = T(k-1) + T(n-k) + \theta(n\ln{n}) \]
که در آن $k$ اندیس عوض یکتای آرایه است.\\
اگر احتمال اینکه عضو یکتای اولی که پیدا میکنیم هر یک از عناصر آرایه باشد را برابر در نظر بگیریم داریم:\\
\[P(k=i) = \frac{1}{n}  \quad \quad  i = (1,2,...,n)\]
پس احتمال اینکه این عضو در یک دوم میانی آرایه باشد برابر با $\frac{1}{2}$ است.
که دقیقا به روابط کوییک سورت میرسیم و قبلا نشان دادیم میانگین زمان اجرای این الگوریتم چگونه محاسبه میشود.
فقط برای آن الگوریتم هزینه ادغام $O(n)$ بود و اگر در حل آن معادله هزینه ادغام را $O(n\ln{n})$ 
در نظر بگیریم میانگین زمان اجرای این الگوریتم برابر با $O(n\ln^2{n})$ است.