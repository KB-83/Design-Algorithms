\problem{}
\subproblem{}
 \[ a = A[n/2] \]
 \[ b = B[n/2] \]
 ابتدا a و b را با هم مقایسه میکنیم
 فرض کنیم b بزرگ تر باشد
 این یعنی $\frac{n}{2}$
 عضو بزرگ تر از b در ارایه A قرار دارد (تمام اعضای بعد از a)
 و در خود ارایه B نیز تمام اعضای بعد از b از آن بزرگ ترند
 پس میانه قطعا کوچک تر مساوی b است و میتوانیم نصفه دوم آرایه 
 ‌B را دور بریزیم
 از طرفی دیگر با همین استدلال میانه حتما بزرگ تر از a است پس حالا میتوانیم
 نیمه اول آرایه A را نیز دور بریزیم
 حالا کافیست میانه را برای دو ارایه هر کدام به طول $\frac{n}{2}$
 حل کنیم (چون تعداد اعضای بزرگ تر و کوچک تر از میانه که دور ریخته شدند برابرند پس میانه دقیقا میانه اعضای باقی مانده است)
 .
 پس رابطه بازگشتی ما به صورت زیر است:
\[ T(n) = 2T(\frac{n}{2}) + O(1)\]
که طبق قضیه مستر برابر با $O(\log{n})$ است.

\subproblem{}
فرض کنیم $m>n$ باشد
ادعا میکنم $\frac{m-n}{2}$ اول و 
$\frac{m-n}{2}$ اخر از آرایه به طول m
میانه نیست و میتوان آنهارا حذف و میانه را برای دو آرایه به طول n حل کرد
با اثبات این ادعا با کمک قسمت قبل $O(\log{a})$
میتوان میانه را پیدا کرد که $a = \min(m,n)$ است.
\proof{}
\[ a = B[\frac{n-m}{2}]\]
\[ b = B[m -\frac{n-m}{2}]\]
تعداد اعضای بزرگ تر از a برابر با $m -\frac{m-n}{2}$ است که برابر است با
$\frac{n+m}{2}$.
از طرفی دیگر میانه کل این $m+n$ عضو برابر با عضو
$\frac{n+m}{2}$ ام است.
پس میانه بزرگ تر مساوی a است.
به همین طریق میانه کوچک تر مساوی b است و میتوان اعضای بزرگ تر از b 
و کوچک تر از a را دور ریخت و ادعا ثابت میشود.


