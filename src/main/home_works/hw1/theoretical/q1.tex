\problem{}
به صورت کلی درباره تمام قسمت ها رابطه ای را که به دست آوردم را اعلام میکنم و سپس با استقرا 
ابتدا نشان میدهم که از O آن رابطه و 
سپس نشان میدهم از $\Omega$ آن رابطه است.

\subproblem{}





% پایه هارو ادد کن
\subproblem{}
ادعا : رابطه از $\Theta(n^2)$ است.\\\\
برهان : رابطه از $O(n^2)$ است. \\\\
به عنوان گام استقرا فرض میکنیم $T(\frac{n - \sqrt{n}}{4}) < C (\frac{n - \sqrt{n}}{4})^2$ حال داریم : \\\\
$T(n) < 8C (\frac{n - \sqrt{n}}{4})^2 + n^2$\\\\
$T(n) < 8C \frac{n^2 +n - n\sqrt{n}}{16} + n^2$\\\\
$T(n) < C \frac{n^2 +n - 2n\sqrt{n}}{2} + n^2$\\\\
$T(n) < C\frac{n^2}{2} + C\frac{n}{2} - Cn\sqrt{n} + n^2$\\\\
$T(n) < (\frac{C}{2}+1)n^2 + C\frac{n}{2} - Cn\sqrt{n}$\\\\
برای $C>2$ و $n>1$ داریم : \\\\
$T(n) < Cn^2$\\\\




\subproblem{}
% wrong
ادعا : رابطه از $\Theta(n^{log3})$ است.\\\\
برهان : رابطه از $O(n^{\log3})$ است. \\\\
به عنوان گام استقرا فرض میکنیم $T(\frac{n}{2}) < C ((\frac{n}{2})^{\log3} - \frac{n}{2})$ حال داریم : \\\\
$T(n) < 3C (\frac{n^{\log3}}{2^{\log3}}) + n$\\\\
$T(n) < Cn^{\log3} + n$  (n>0)\\\\
$T(n) < C(n)^{\log3}$\\\\\\

برهان : رابطه از $\Omega(n^{\log3})$ است. \\\\
به عنوان گام استقرا فرض میکنیم $T(\frac{n}{2}) > C (\frac{n}{2})^{\log3}$ حال داریم : \\\\
$T(n) > 3C (\frac{n^{\log3}}{2^{\log3}}) + n$\\\\
$T(n) > Cn^{\log3} + n$  \quad $(n>0)$\\\\
$T(n) > C(n)^{\log3}$\\\\\\

\subproblem{}
\subproblem{}
\subproblem{}
\subproblem{}
