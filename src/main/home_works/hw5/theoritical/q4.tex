\problem{}
الگوریتم جریان بیشینه به ما بیشترین جریان گذرنده بدون یال تکراری را میدهد.

اما ما میخواهیم راس ها نیز تکراری نباشند.
ایده جالب این است که راس ها را به چشم یال نگاه کنیم و ظرفیت یک به آنها بدهیم که مثلا
اگر راسی میتوانست دو واحد جریان عبور دهد حال به یالی با یک واحد جریان تبدیل شده
و حداکثر یک واحد جریان عبور میدهد.
پس هر راسی که مینیمم درجه ورودی و خروجی آن بیشتر از یک بود را به عنوان دو راس 
i و o در نظر میگیرم و یال جهت داری
از i به o رسم میکنم و یال هایی که به این راس وارد شده بودند را به 
i وارد میکنم 
و همچنین آنهایی که از آن خارج شده بودند را از o خارج میکنم.
حالا الگوریتم $\text{\LR{max flow}}$ را روی این گراف جدید بین دو راس s و t اجرا میکنیم
به این صورت که به یال ظرفیت یک میدهم و اگر گراف جهت دار نبود نیز هر یال را به صورت 
دو یال جهت دار میکشم.
ادعا میکنم جریان بیشینه ای که از این گراف رد میشود  با تعداد مسیر های
راس مجزا بین s و t برابر است.\\\\
اثبات ادعا:\\\\
جریان به اندازه بیشترین تعداد مسیر های راس مجزا به وضوح در گراف جدید قابل عبور دادن است 
زیرا در هر مسیر جریان $1$ واحد جاری میشود
و به هر راس مسیر دقیقا جریان یک وارد 
و جریان یک از آن خارج میشود چون در مجموع
مسیر ها هیچ راس تکراری وجود ندارد.
پس چنین جریانی قابل عبور دادن است.\\
حال کافیست نشان دهم که جریان بیشینه عبوری همان جریان بیشینه برای مسیر های راس مجزا است
یعنی باید نشان دهم که ممکن نیست این جریان
از دو مسیری که راس مشترک دارند عبور کند.
فرض کنیم راس مشترک آنها راس a است .
حالا در گراف جدیدی که ساختم $a_1$ و $a_2$
 را داریم که ماکسیمم جریان بین آنها یک است
 پس اگر بخواهیم هر دوی این مسیر ها را داشته باشیم
 باید از راس a دو واحد جریان عبور کند
 که اکنون به علت پایستگی جریان ماکسیمم جریان عبوری از $a_1$ و $a_2$
 برابر $1$ است و این نشان میدهد هیج دو مسیر با راس مشترکی در 
گراف اصلی در جریان بیشینه این گراف کمکی انتخاب نمیشوند و اثبات تکمیل میشود.