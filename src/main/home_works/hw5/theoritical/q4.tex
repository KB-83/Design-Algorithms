\problem{}
الگوریتم جریان بیشینه به ما بیشترین جریان گذرنده بدون یال تکراری را میدهد.

اما ما میخواهیم راس ها نیز تکراری نباشند.
ایده جالب این است که راس ها را به چشم یال نگاه کنیم و ظرفیت یک به آنها بدهیم که مثلا
اگر راسی میتوانست دو واحد جریان عبور دهد حال به یالی با یک واحد جریان تبدیل شده
و حداکثر یک واحد جریان عبور میدهد.
پس هر راسی که مینیمم درجه ورودی و خروجی آن بیشتر از یک بود را به عنوان دو راس 
i و o در نظر میگیرم و یال جهت داری
از i به o رسم میکنم و یال هایی که به این راس وارد شده بودند را به 
i وارد میکنم 
و همچنین آنهایی که از آن خارج شده بودند را از o خارج میکنم.
حالا الگوریتم max flow را روی این گراف جدید بین دو راس s و t اجرا میکنیم
به این صورت که به یال ظرفیت یک میدهم و اگر گراف جهت دار نبود نیز هر یال را به صورت 
دو یال جهت دار میکشم.
ادعا میکنم جریان بیشینه ای که از این گراف رد میشود برابر است با تعداد مسیر های
راس مجزا بین s و t.\\
اثبات ادعا:\\
