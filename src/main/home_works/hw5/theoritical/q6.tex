\problem{}
% ابتدا گراف داده شده را به صورت توپو
برای حل این سوال گراف $G^{\prime}$
.را به صورتی که در صورت سوال گفته شده است تشکیل میدهم
سپس به هر یال وزن ۱ نسبت میدهم.
حالا یک جریان از $x_0$
به اندازه n
وارد گراف کرده و در نهایت الگوریتم مکس فلو به یک سری از یال ها عدد یک را نسبت
میدهد.
گراف S را به صورت زیر میسازم.\\
\[
    V_s = V_g\\
\]
\[
    (i,j) \in E_g \quad \text{iff} \quad f_{(x_i,y_j)} = 1\\
\]
\[
    \text{\LR{for}} \quad  1\leq i,j\leq n
\]
که:  $f_e = \text{جریان نسبت داده شده به یال e}$ \\\\
ادعا میکنم که این گراف کمترین پوشش مسیری برای گراف $G$ است.\\\\
اثبات:\\\\
الگوریتم جریان بیشینه در گراف $G^{\prime}$ 
بیشترین تعداد یال های ممکن را پیدا میکند که پوشش مسیری راس مجزا به ما میدهند.\\
دلیل راس مجزا بودن این است که اگر$(x_i,y_j)$ انتخاب شد
دیگر $(x_i,y_z)$ یا $(x_t,y_j)$ نمیتوانند انتخاب شوند.
چون جریان ورودی و خروحی یک واحد است و پایستگی جریان داریم.
و از طرفی دیگر گراف دور ندارد پس تضمین میشود که مجزا بودن راس ها را داریم.\\
حالا کافیست نشان دهیم
کمترین پوشش مسیری همان پوششی است که بیشترین تعداد یال را دارد.
\\
یک پوشش مسیری با تعداد مسیر مشخص میشود که تعداد یال های هر مسیر
برابر است با:\\
\[\text{تعداد راس های آن مسیر } -1\]
\\
پس تعداد کل یال های این پوشش برابر است با 
\[ \text{تعداد راس ها} - r\]
که $r$ تعداد مسیر هاست.
پس هرچه $r$ کمتر شود تعداد یال ها بیشتر و همچنین
هرچه تعداد یال ها بیشتر یعنی تعداد مسیر ها کمتر بوده.
پس پوشش مسیری با بیشترین یال ممکن همان پوشش مسیری با کمترین تعداد مسیر است.
پس الگوریتم ثابت میشود.
\\\\
تحلیل زمانی:\\\\
تشکیل گراف $S$ از روی $f_e$
های بدست آمده برای یال ها چند جمله ایست و همچنین الگوریتم بیشینه جریان نیز
چند جمله ای حل میشود پس
این الگوریتم چند جمله ای است.