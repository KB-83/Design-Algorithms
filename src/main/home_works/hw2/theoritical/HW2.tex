\documentclass[12pt]{article}
\usepackage{HomeWorkTemplate}
\usepackage{circuitikz}
\usepackage[shortlabels]{enumitem}
\usepackage{fontspec}
\usepackage{tikz}
\usepackage{xepersian}
\usepackage{graphicx}
\usepackage{float}
\usepackage{amsmath}
\usepackage{import}


\usetikzlibrary{arrows,automata}
\usetikzlibrary{circuits.logic.US}
\settextfont{XB Niloofar}
\newcounter{problemcounter}
\newcounter{subproblemcounter}
\setcounter{problemcounter}{1}
\setcounter{subproblemcounter}{1}
\newcommand{\problem}[1]
{
	\subsection*{
		پرسش
		\arabic{problemcounter} 
		\stepcounter{problemcounter}
		\setcounter{subproblemcounter}{1}
		#1
	}
}
\newcommand{\subproblem}{
	\textbf{\harfi{subproblemcounter})}\stepcounter{subproblemcounter}
}



\begin{document}

\handout
{آنالیز الگوریتم ها}
{تمرین سری دو}
{کژال باغستانی}
{۴۰۱۱۰۰۰۷۱}

فرض ۱: الگوریتمی داریم که در زمان خطی مولفه های قویا همبند را در گراف جهت دار پیدا میکند.\\
فرض ۲: الگوریتمی داریم که در زمان خطی topological sort را برای یک dag پیدا میکند.
در ادامه این الگوریتم هارا ارائه میدهم.
% have to be compeleted.
\problem{}
برای حل این سوال آرایه دو بعدی dp[][] را در نظر میگیریم
و به ترتیب طول زیر رشته ها این آرایه را پر میکنیم به چه صورت؟
به این صورت که dp[i][j]
 برابر است با طول بلند ترین زیر آرایه پالیندروم 
 با شروع از کارکتر i ام رشته و پایان در کاراکتر j ام رشته.
حالا قرار میدهیم :\\
\[
    dp[i][j] = max(dp[i,j-1],dp[i+1][j],dp[i+1][j-1]+\delta_{i,j})    
\]
و داریم:\\
\[
    \delta_{i,j}=
    \begin{cases}
        2 \quad  s[i]=s[j]\\
        0 \quad  \text{else}
    \end{cases}
\]
و اگر زیر فضای مساله ها را ماتریس دو بعدی در نظر بگیریم که ستون های آن نشان دهنده i و
سطر های آن نشان دهنده j است
مساله های اولیه عناصر روی قطر هستند
که همگی یک و در ادامه قطر بعدی را نیز باید به عنوان زیر مساله پر کرد که اگر آن دو حرف با هم برابر بودند
مقدار آن عنصر ۲ و در غیر این صورت ۱ است.
و در ادامه عناصر هر قطر از روی عناصر قطر قبل ساخته میشود.
حالا که طول بلند ترین زیر رشته را پیدا کردیم میتوانیم از جدول برگردیم به عقب و خود زیر آرایه
را پیدا کنیم.
به چه صورت؟\\
در هر قدم وقتی ماکسیمم میگریم یکی از آپشن ها به عنوان ماکسیمم انتخاب شده
حالا اگر یک پوینتر قرار دهیم که از خانه ای که آن را پر کرده به خانه ای که به عنوان ماکسیمم انتخاب
شده اشاره کنیم و اگر خانه پر شده, ماکسیمم خود را وقتی میگرفت که $\delta_{ij}$
برابر با دو بود (یعنی باعث میشد دو حرف به سرو و ته رشته اضافه کنیم)
آن خانه را ستاره دار میکنیم.
 در نتیجه
پس از پر کردن جدول
از خانه اخر که شامل کل رشته میشود شروع کرده
($dp[1][size(string)]$)
و پوینتر هارا دنبال میکنیم
در هر مرحله اگر خانه ای که به آن رسیدیم ستاره دار بود 
ما نیز به سرو و ته رشته ای که داریم تشکیل میدهیم
آن حرف را اضافه میکنیم.
و در آخر رشته خروجی ساخته شده همان زیر آرایه مورد نظر است.
\\\\
تحلیل زمان اجرا:\\\\
پر کردن آرایه دوبعدی ما $O(n^2)$ زمان میبرد و همچنین
بازگشت از خانه اخر به عقب و ساختن رشته نیز $O(n)$ است.
پس کل زمان اجرای الگوریتم برابر است با $O(n^2)$.

\problem{}
ابتدا فرض کنید که یک یال در دور همیلتونی ما هست.
با حذف آن یال باید بتوانیم یک مسیر همیلتونی به ترتیب راس های قبل فقط با شروع و پایان از
دو سر این یال داشته باشیم.
پس ابتدا یک یال دلخواه در نظر میگیریم.
سپس آن یال را حذف کرده و دو راس دو سر آن را u و v مینامیم.
حالا تمام حالت های انتخاب دو راس از راس های مجاور این یال ها را در نظر میگیریم . 
یال بین این دو راس و بقیه راس های مجاور را حذف میکنیم.
یعنی تمام حالاتی که درجه راس های u و v ۱ میشود در نظر میگیریم.
حالا با استفاده از الگوریتم A چک میکنیم که گراف باقیمانده مسیر همیلتونی دارد یا خیر؟
در نهایت به ازای همه یال ها و همه حالت ها اگر حالتی وجود داشت که جواب بله بود الگوریتم
بله خروجی میدهد. و در غیر این صورت جواب خیر است.

اثبات درستی:\\
تحلیل زمانی:\\

\problem{}
ابتدا یک گراف کمکی میسازم که راس های آن راس های علامت دار گراف داده شده هستند.
سپس
از یک راس علامت دار دلخواه شروع کرده و dfs را اجرا میکنم تا به راس
علامت دار دیگری برسم
سپس بین این دو راس علامت دار یک یال به وزن فاصله این دو راس اضافه میکنم و همچنین یال های پیمایش شده را
در dfs حذف میکنم
حالا یک راس علامت دار دیگر انتخاب کرده
و در صورتی که مسیری از آن وجود داشت آن را دنبال کرده تا به راس 
علامت دار دیگری برسم یا در dfs به خود آن راس برگردم و تمام یال های مسیر
را حذف کنم.
در نهایت این کار را انقدر ادامه میدهم تا گراف به گراف بدون یال تبدیل شود.
حالا در گراف به دست آمده وزن یال ها را در هم ضرب و به عنوان جواب سوال خروجی میدهم.
\\\\
اثبات درستی:\\\\
چونکه گراف داده شده یک درخت است بین هر دو راس علامت دار دقیقا یک مسیر
وجود دارد حالا من در گراف کمکی به صورتی دارم راس های علامت دار را به عنوان مولفه
همبندی در نظر گرفته و بین دو راس علامت داری که یک مسیر مستقیم وجود دارد(مسیری که در آن راس علامت دار دیگری نیست)
یک یال به وزن طول آن مسیر در نظر میگیرم.
و درخت بودن گراف اولیه باعث میشود گراف
علامت دار جدید نیز درخت باشد.
حالا دو راس علامت دار مستقیم را در نظر بگیرید که بین آنها در گراف کمکی یال هست
مثلا وزن این یال r است.
این یعنی r یال در گراف اصلی بین این دو راس بوده
حالا ما یکی از این r یال را انتخاب و حذف کرده
و راس های که متصل به راس علامت دار اول میمانند
در مولفه همبندی آن قرار میگیرند و بقیه در مولفه همبندی
راس دوم قرار میگیرند.
با توجه به این که راس های این مسیر باید حتما در یکی از این دو ملفه همبندی قرار میگرفتند
و از یک جایی ما باید برش میزدیم و دقیقا r حالت داریم
پس با ضرب تمام این وزن ها در هم تمام حالت ها به پوشش داده میشود.
(دقت کنید که گراف کمکی صرفا برای درک بهتر مولفه های همبندی بوده و کمک در اثبات درستی وگرنه میتوانستیم
مستقیما اعداد بدست آمده را در هم ضرب کنیم).
\\\\
تحلیل زمانی:\\\\
الگوریتم dfs تعمیم یافته ای که اجرا میکنیم در مجموع
به ازای هر یال $O(1)$ عملیات انجام میدهد.
به صورت دقیق تر یال های هر مسیر را نگه میدارد تا به راس علامت دار اولیه برسد
و یا به راس علامت دار دیگری سپس آن یال های نگه داشته شده را از گراف حذف میکند.
که این برابر $O(\text{طول آن مسیر})$
زمان میبرد.
و در نهایت مجموع همه مسیر ها میشود $O(m+n)$
که چون گراف داده شده درخت است برابر است با 
$O(n)$ و در نهایت گراف کمکی در بدترین حالت n-1 یال دارد
که پیمایش روی آنها نیز $O(n)$ است.
پس زمان اجرایی کل الگوریتم برابر است با $O(n)$.
\problem{}
% need to be compeleted and edited
هر جا سخن از گراف جهت دار بدون دور است , نام ترتیب توپولوژیکال است که میدرخشد.\\
مساله را با تکنیک برنامه ریزی پویا حل میکنیم به این صورت که 
پس از بدست آوردن topological sort 
به ترتیب توپولوژیکال روی راس ها حرکت کرده و طول بلند ترین مسیر به i را قرار میدهیم:\\
\[
    longest[i] = max(longest[j]+1)\quad \text{\LR{for all j that we have (j,i) in E}}
\]
\\
اثبات درستی:\\
 چون در ترتیب توپولوژیکال اگر  (j,i) عضو E
باشد آنگاه حتما j قبل از i در این ترتیب آمده
پس با حرکت کردن روی راس ها به ترتیب توپولوژیکال و داشتن longest  برای تمام j های 
کوچک تر از i حتما برای تمام j ها که (j,i) عضو E است
مقدار longest را داریم.
در نتیجه این مساله حل میشود.
و البته اولین زیر مساله برای راس اول هست \LR{longest[u] = $0$}
در نهایت روی آرایه longest حرکت کرده و max میگیریم.\\\\
تحلیل زمان اجرا:\\
اردر توپولوژیکال سورت همان اردر dfs است. $O(n+m)$
 و همچینین پیمایش برای پر کردن آرایه longest نیز در مجموع روی هر راس پیمایش کرده
 و هر یال یک بار در نظر گرفته میشود که در مجموع میشود $O(m+n)$ 
 و همچینین ماکسیمم گرفتن نیز $O(n)$ است.
 پس در کل
 زمان اجرای کل الگوریتم برابر است با $O(n+m)$.
\problem{}
برای حل این سوال ابتدا سه بار الگوریتم دایکسترا را از
راس های w و v و u
اجرا کرده و به ازای هر راس
مینیمم فاصله آن تا هر سه راس مد نظر را داریم.\\
حالا به صورت زیر عمل میکنیم:\\
روی تمام راس ها پیمایش کرده و مجموع فاصله ها از u و v و w را با هم جمع
کرده و از آنها مینیمم میگیریم.
سپس راسی که به عنوان مینیمم مجموع فاصله های آن خروجی داده میشود را 
بر میداریم و دوباره دایکسترا را روی آن اجرا میکنیم این بار مقادیر pre را هم برای
مشخص کردن مسیر نگه میداریم.
و سپس یال های این مسیر ها برای آن راس تا 
راس های u و 
سپس v و
سپس w 
به زیر گراف خروجی اضافه میکنیم.\\\\
اثبات درستی:\\\\
ممکن است برای هر راس عدد داده شده دقیقا مجموع فاصله آن راس از u و v و w نباشد 
(مثلا مسیر آن راس به u زیر گرافی از مسیر آن راس به v باشد).
اما حتما یک راس وجود دارد که دقیقا با جمع کردن این سه مقدار
 اندازه مسیر مینیمم را به ما میدهد.\\
فرض کنیم زیرگرافی با مجموع وزن مینیمم داریم.
قطعا برای یکی از راس های این گراف سه مسیر آن راس به u و v و w 
هیچ اشتراکی با هم ندارند و وقتی که الگوریتم فاصله این سه راس از راس مورد نظر در
الگوریتم محاسبه میشود وزن این زیر گراف اضافه شده و این زیر گراف یا زیر گرافی هم وزن
آن به عنوان خروجی انتخاب میشوند.\\\\\\
تحلیل زمان اجرا:\\\\
ابتدا سه بار دایکسترا را اجرا میکنیم که $O(m\log{n})$.\\
سپس روی راس ها پیمایش کرده و سه عدد را جمع میکنیم$O(n)$.\\
در نهایت ماکسیمم اعداد بدست آمده در مرحله قبل را محاسبه میکنیم $O(n)$.
و در آخر یک بار دیگر دایکسترا را اجرا میکنیم که در مرحله قبل محاسبه شده و یال های بدست
آمده را نیز درج میکنیم.\\
 که در مجموع $O((m+n)\log{n})$ است.\\
\problem{}
برای اثبات این الگوریتم کافیست دو چیز را ثابت کنیم که به صورت دو ادعا
در زیر ارائه میکنم.\\
% ادعا ۱:\\
% در درخت BFS بدست آمده فاصله بین دو رأس i و j که
% هیچ کدام در سطح آخر نیستند نمیتواند قطر درخت باشد.\\
% اثبات:\\

در ابتدا به این نکته توجه میکنیم که در درخت هر گاه مسیری بین دو رأس داریم
آن مسیر تنها مسیر بین آن دو رأس است.
(زیرا در غیر این صورت دور تشکیل میشود)
درنتیجه مسیری که بین دو رأس i و j  در درخت BFS
وجود دارد ابتدا از i به ریشه میرویم
و اگر j در این مسیر نبود
از j به سمت ریشه میرویم و اولین جایی که به مسیر از i به ریشه برخورد کردیم
میتوانیم به سمت i برویم.
پس هر مسیری یک بخش بالا رونده به سمت ریشه و یک بخش پایین رونده به سمت
رأس مورد نظر دارد.
و نکته دوم قابل توجه این است که قطر درخت هرگز نمیتواند بین i و j ای باشد
که یکی در مسیر دیگری به ریشه است. زیرا فاصله آن رأس تا ریشه بیشتر از فاصله
آن تا رأس دیگر میشود.\\



ادعا ۱:\\
اگر قطر گراف طول D داشته باشد
هر رأس را به عنوان ریشه درخت BFS انتخاب کنیم
حتما یک سر قطر در سطح آخر می افتد.\\
اثبات:\\
رأس v را به عنوان ریشه درخت BFS در نظر میگیریم.
سر اول قطر از آن فاصله $H$ و سر دوم قطر از آن فاصله
$D - H$ دارد.
در این صورت ارتفاع درخت BFS برابر با $max(H,D-H)$ میشود.
زیرا اگر بیشتر شود رأسی به فاصله بیشتر از این دو, از ریشه درخت وجود دارد
که این یعنی آن رأس به عنوان سر دیگر قطر شناخته میشود.\\



% حالا به سراغ اثبات میرویم:\\
% اگر یکی از این دو برگ نباشد به وضوح میتوان از آن به طبقه
% پایین تر رفت و یک واحد به بخش پایین رونده مسیری که قبلا داشتیم یعنی قطر درخت افزود.\\
% حالا اگر هر دو رأس i و j 
% برگ باشند و هیچ کدام در سطح اخر نباشند
% اگر قسمت بالا رونده این مسیر را طی کنیم و برای قسمت پایین رونده
% از جد مشترک آنها به سمت رأس با ارتفاع بیشتر برویم طول مسیر بیشتر و قطر
% درخت زیاد میشود.
% در نهایت اگر جد مشترک i و j نیز به رأس با ارتفاع بیشتر راهی به سمت پایین نداشت
% پس قطعا هیچ کدام از عناصر سطح اخر با i جد مشترک
% پایین تر از جد مشترک i و j ندارند در نتیجه میتوان
% یکی از آنهارا انتخاب کرد و مسیر بین آن و i قطعا بلند تر از مسیر
% بین i و j میشود زیرا هم مسیر بالا رونده که رسیدن به جد مشترک است
% طولانی تر
% و هم مسیر پایین رونده طولانی تر میشود.
% پس ادعا ثابت میشود.\\
ادعا ۲:\\
اگر در درخت BFS یک راس از سطح آخر انتخاب کنیم
و BFS را روی آن اجرا کنیم
مستقل از اینکه آن کدام رأس است عدد یکسانی به عنوان قطر به ما میدهد.\\
اثبات:\\
فرض کنیم $a_1,a_2,..,a_n$ رأس های سطح آخر ما هستند.
اگر $a_i$ راس i را به عنوان سر دیگر قطر خروجی دهد 
اگر $a_j$ را نیز در نظر بگیریم .
یا جد مشترک آن با i همان جد مشترک $a_i$ با i است
که در این صورت همان فاصله را با i دارد.
و یا جد مشترک آن با i متفاوت از جد مشترک $a_i$ و i است که در این
صورت اگر در نظر بگیریم جد مشترک $a_j$ و i پایین تر از جد مشترک
$a_i$ و i است.
اگر i در سطح آخر نباشد آنگاه از جد مشترک i و $a_j$  بجای رفتن 
به i به $a_j$ میرویم.
که این یعنی طول قطر برای $a_i$ بیشتر میشد.
پس اگر جد مشترک متفاوت داشتیم نیز فاصله $a_j$ و $a_i$
با هم را به عنوان خروجی برای $a_j$ در نظر میگیریم.
(زیرا برابر با فاصله $a_i$ و i بود.)
حالا کافیست نشان دهیم که هیج کدام از $a_i$ ها نمیتوانند
عدد بزرگ تری از دیگری به قطر نسبت بدهند. 
این بخش نیز به وضوح معلوم است چون نشان دادیم
هر عددی که از هر کدام از $a_i$ ها بگیریم به سادگی میتوان
به $a_j$ دیگر نیز راسی نسبت داد که همان عدد را بدهد.
پس همگی مساوی و ماکسیمم نداریم.
\\


% و در آخر اثبات مستقل از این بود که کدام رأس به عنوان ریشه درخت BFS انتخاب
% میشود اما برای تکمیل آن نیز ادعا زیر را ثابت میکنم.\\


% ادعا ۳:\\
% اگر قطر گراف طول D داشته باشد
% هر رأس را به عنوان ریشه درخت BFS انتخاب کنیم
% حتما یک سر قطر در سطح آخر می افتد.\\
% اثبات:\\
% رأس v را به عنوان ریشه درخت BFS در نظر میگیریم.
% سر اول قطر از آن فاصله $H$ و سر دوم قطر از آن فاصله
% $D - H$ دارد.
% در این صورت ارتفاع درخت BFS برابر با $max(H,D-H)$ میشود.
% زیرا اگر بیشتر شود رأسی به فاصله بیشتر از این دو, از ریشه درخت وجود دارد
% که این یعنی آن رأس به عنوان سر دیگر قطر شناخته میشود.\\
و اثبات کامل میشود.


\end{document}