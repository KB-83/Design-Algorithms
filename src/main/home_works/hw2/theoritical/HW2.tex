\documentclass[12pt]{article}
\usepackage{/Users/kajal/Documents/Design-Algorithms-Course/resources/latex/template/HomeWorkTemplate}
\usepackage{circuitikz}
\usepackage[shortlabels]{enumitem}
\usepackage{fontspec}
\usepackage{tikz}
\usepackage{xepersian}
\usepackage{graphicx}
\usepackage{float}
\usepackage{amsmath}
\usepackage{import}


\usetikzlibrary{arrows,automata}
\usetikzlibrary{circuits.logic.US}
\settextfont{XB Niloofar}
\newcounter{problemcounter}
\newcounter{subproblemcounter}
\setcounter{problemcounter}{1}
\setcounter{subproblemcounter}{1}
\newcommand{\problem}[1]
{
	\subsection*{
		پرسش
		\arabic{problemcounter} 
		\stepcounter{problemcounter}
		\setcounter{subproblemcounter}{1}
		#1
	}
}
\newcommand{\subproblem}{
	\textbf{\harfi{subproblemcounter})}\stepcounter{subproblemcounter}
}



\begin{document}

\handout
{آنالیز الگوریتم ها}
{تمرین سری دو}
{کژال باغستانی}
{۴۰۱۱۰۰۰۷۱}

% فرض ۱: الگوریتمی داریم که در زمان خطی مولفه های قویا همبند را در گراف جهت دار پیدا میکند.\\
% فرض ۲: الگوریتمی داریم که در زمان خطی topological sort را برای یک dag پیدا میکند.
% در ادامه این الگوریتم هارا ارائه میدهم.
\problem{}
به صورت کلی درباره تمام قسمت ها رابطه ای را که به دست آوردم را اعلام میکنم و سپس با استقرا 
ابتدا نشان میدهم که از O آن رابطه و 
سپس نشان میدهم از $\Omega$ آن رابطه است.

\subproblem{}





% پایه هارو ادد کن
\subproblem{}
ادعا : رابطه از $\Theta(n^2)$ است.\\\\
برهان : رابطه از $O(n^2)$ است. \\\\
به عنوان گام استقرا فرض میکنیم $T(\frac{n - \sqrt{n}}{4}) < C (\frac{n - \sqrt{n}}{4})^2$ حال داریم : \\\\
$T(n) < 8C (\frac{n - \sqrt{n}}{4})^2 + n^2$\\\\
$T(n) < 8C \frac{n^2 +n - n\sqrt{n}}{16} + n^2$\\\\
$T(n) < C \frac{n^2 +n - 2n\sqrt{n}}{2} + n^2$\\\\
$T(n) < C\frac{n^2}{2} + C\frac{n}{2} - Cn\sqrt{n} + n^2$\\\\
$T(n) < (\frac{C}{2}+1)n^2 + C\frac{n}{2} - Cn\sqrt{n}$\\\\
برای $C>2$ و $n>1$ داریم : \\\\
$T(n) < Cn^2$\\\\




\subproblem{}
% wrong
ادعا : رابطه از $\Theta(n^{log3})$ است.\\\\
برهان : رابطه از $O(n^{\log3})$ است. \\\\
به عنوان گام استقرا فرض میکنیم $T(\frac{n}{2}) < C ((\frac{n}{2})^{\log3} - \frac{n}{2})$ حال داریم : \\\\
$T(n) < 3C (\frac{n^{\log3}}{2^{\log3}}) + n$\\\\
$T(n) < Cn^{\log3} + n$  (n>0)\\\\
$T(n) < C(n)^{\log3}$\\\\\\

برهان : رابطه از $\Omega(n^{\log3})$ است. \\\\
به عنوان گام استقرا فرض میکنیم $T(\frac{n}{2}) > C (\frac{n}{2})^{\log3}$ حال داریم : \\\\
$T(n) > 3C (\frac{n^{\log3}}{2^{\log3}}) + n$\\\\
$T(n) > Cn^{\log3} + n$  \quad $(n>0)$\\\\
$T(n) > C(n)^{\log3}$\\\\\\

\subproblem{}
\subproblem{}
\subproblem{}
\subproblem{}

\problem{}
\problem{}
ابتدا از روی لیست 
$\text{f}_i$
 ها گرافی جهت دار که راس های آن
کارمندان و یال های آن به صورت جهت دار از هر کارمند به کارمندی است که باید
به آن خبر رسانی کند تشکیل میدهیم.
سپس با الگوریتم $1$ مولفه های قویا همبند را $O(|V|+|E|)$ پیدا میکنیم که در اینجا
تعداد یال ها دقیقا با تعداد راس ها برابر است (زیرا هر کارمندی دقیقا به یک کارمند یال جهت دار دارد.)
پس $O(n)$.\\
ادعا:\\
هر شخصی خبری که میشنود به همه افراد منتقل میشود اگر و تنها اگر تمام افراد در یک مولفه
قویا همبند قرار گیرند.\\
اثبات :\\
اگر تمام افراد در یک مولفه قویا همبند قرار گیرند به وضوح از هر راس (کارمند)
به راس های دیگر مسیری جهت دار هست پس خبر به خوبی پخش میشود.\\
حال میخواهم نشان دهم اگر تمام افراد در یک مولفه قویا همبند نباشند 
فردی وجود دارد که اگر خبر به او برسد همه مطلع نمیشوند.
می دانیم هر dag حداقل یک sink دارد پس
اگر خبر به یکی از افراد در مولفه ای که sink است برسد
خبر فقط در همان مولفه میماند و افراد در مولفه های دیگر از آن خبر
مطلع نمیشوند پس ادعا ثابت میشود.\\
ادعا:\\
کمترین یال مورد نیاز برای همبند کردن گراف dag
برابر است با $max(a,b)$ که a تعداد رأس ها
با درجه ورودی صفر و b تعداد رأس ها با درجه خروجی صفر است.\\
اثبات:\\
اگر از هر رأس به رأس دیگر راه وجود داشته باشد حتما تمامی رأس ها
حداقل یک یال ورودی و خروجی دارند.
پس حداقل به تعداد $max(a,b)$ نیاز داریم.
حالا الگوریتمی ارائه میدهم که با این تعداد 
dag را تبدیل به گراف قویا همبند کند.
توجه کنید که راس هایی که یا درجه ورودی ندارند یا درجه خروجی یا 
سینک هستند و یا سورس .
ابتدا مشخص میکنیم که a بیشتر است یا b 
فرض کنیم که a>b داریم:\\
سورس ها بیشتر از سینک ها در نتیجه هر سینک را به سورس بعدی آن وصل میکنیم
برای ایجاد دور همیلتونی و اگر دو یا چند سورس بعد از سینک مورد نظر بود
سینک مورد نظر را به هر دو وصل میکنیم
به صورت شهودی مانند قطعه های پازل گراف های dag را میچینیم و سینک های قطعه قبلی
را به سورس های قطعه بعدی متصل میکنیم.
با اینکار به وضوح دور همیلتونی تشکیل میشود و تعداد یال های مورد نیاز به تعداد
سورس هاست.
اگر سینک ها بیشتر بود یعنی b>a نیز به صورت مشابه عمل میکنیم و الگوریتمی ارائه دادیم که
با $max(a,b)$ دور همیلتونی ایجاد میکند و اثبات کامل میشود.
\problem{}
% need to be compeleted and edited
هر جا سخن از گراف جهت دار بدون دور است , نام ترتیب توپولوژیکال است که میدرخشد.\\
مساله را با تکنیک برنامه ریزی پویا حل میکنیم به این صورت که 
پس از بدست آوردن topological sort 
به ترتیب توپولوژیکال روی راس ها حرکت کرده و طول بلند ترین مسیر به i را قرار میدهیم:\\
\[
    longest[i] = max(longest[j]+1)\quad \text{\LR{for all j that we have (j,i) in E}}
\]
\\
اثبات درستی:\\
 چون در ترتیب توپولوژیکال اگر  (j,i) عضو E
باشد آنگاه حتما j قبل از i در این ترتیب آمده
پس با حرکت کردن روی راس ها به ترتیب توپولوژیکال و داشتن longest  برای تمام j های 
کوچک تر از i حتما برای تمام j ها که (j,i) عضو E است
مقدار longest را داریم.
در نتیجه این مساله حل میشود.
و البته اولین زیر مساله برای راس اول هست \LR{longest[u] = $0$}
در نهایت روی آرایه longest حرکت کرده و max میگیریم.\\\\
تحلیل زمان اجرا:\\
اردر توپولوژیکال سورت همان اردر dfs است. $O(n+m)$
 و همچینین پیمایش برای پر کردن آرایه longest نیز در مجموع روی هر راس پیمایش کرده
 و هر یال یک بار در نظر گرفته میشود که در مجموع میشود $O(m+n)$ 
 و همچینین ماکسیمم گرفتن نیز $O(n)$ است.
 پس در کل
 زمان اجرای کل الگوریتم برابر است با $O(n+m)$.
\problem{}
با روش تقسیم و حل این سوال را حل خواهم کرد و ایده اصلی سوال تقسیم کردن حول
عوض یکتا در هر مرحله است.\\
ابتدا در نظر میگیریم که در هر آرایه دلخواه اگر بخواهیم تمام زیرآرایه ها دارای
عضو یکتا باشند خود آرایه اصلی که جز زیرآرایه هاست نیز دارای عوض یکتاست.
پس در هر آرایه واجد شرایط حداقل یک عضو یکتا وجود دارد و ما تقسیم و حل را
حول آن عضو انجام میدهیم.\\
اگر تمام زیر آرایه های آرایه ی سمت راست عضو یکتا دارای عضو یکتا باشند 
و همچنین تمام زیر آرایه های آرایه سمت چپ عضو یکتا نیز دارای عضو یکتا باشند
به وضوح تمام زیر آرایه هایی که یک قسمت از آنها در زیر آرایه سمت چپ
و قسمت دیگر آنها در زیر آرایه سمت راست هستند به دلیل اینکه شامل عضو یکتا میشوند
حتما دارای عضو یکتا هستند.(که همان عضو یکتایی است که حول آن تقسیم و حل انجام دادیم)
پس کافیست عضو یکتا را بیابیم که با مرتب سازی آرایه ($O(n\ln{n})$)
و پیمایش روی آن ($O(n)$)
میتوان عضو یکتا را پیدا کرد.
پس رابطه بازگشتی ما به صورت زیر است.\\
\[ T(n) = T(k-1) + T(n-k) + \Theta(n\ln{n}) \]
که در آن $k$ اندیس عوض یکتای آرایه است.\\
اگر احتمال اینکه عضو یکتای اولی که پیدا میکنیم هر یک از عناصر آرایه باشد را برابر در نظر بگیریم داریم:\\
\[P(k=i) = \frac{1}{n}  \quad \quad  i = (1,2,...,n)\]
پس احتمال اینکه این عضو در یک دوم میانی آرایه باشد برابر با $\frac{1}{2}$ است.
که دقیقا به روابط کوییک سورت میرسیم و قبلا نشان دادیم میانگین زمان اجرای این الگوریتم چگونه محاسبه میشود.
فقط برای آن الگوریتم هزینه ادغام $O(n)$ بود و اگر در حل آن معادله هزینه ادغام را $O(n\ln{n})$ 
در نظر بگیریم میانگین زمان اجرای این الگوریتم برابر با $O(n\ln^2{n})$ است.
\problem{}
% ابتدا گراف داده شده را به صورت توپو
برای حل این سوال گراف $G^{\prime}$
.را به صورتی که در صورت سوال گفته شده است تشکیل میدهم
سپس به هر یال وزن ۱ نسبت میدهم.
حالا یک جریان از $x_0$
به اندازه n
وارد گراف کرده و در نهایت الگوریتم مکس فلو به یک سری از یال ها عدد یک را نسبت
میدهد.
گراف S را به صورت زیر میسازم.\\
\[
    V_s = V_g\\
\]
\[
    (i,j) \in E_g \quad \text{iff} \quad f_{(x_i,y_j)} = 1\\
\]
\[
    \text{\LR{for}} \quad  1\leq i,j\leq n
\]
که:  $f_e = \text{جریان نسبت داده شده به یال e}$ \\\\
ادعا میکنم که این گراف کمترین پوشش مسیری برای گراف $G$ است.\\\\
اثبات:\\\\
الگوریتم جریان بیشینه در گراف $G^{\prime}$ 
بیشترین تعداد یال های ممکن را پیدا میکند که پوشش مسیری راس مجزا به ما میدهند.\\
دلیل راس مجزا بودن این است که اگر$(x_i,y_j)$ انتخاب شد
دیگر $(x_i,y_z)$ یا $(x_t,y_j)$ نمیتوانند انتخاب شوند.
چون جریان ورودی و خروحی یک واحد است و پایستگی جریان داریم.
و از طرفی دیگر گراف دور ندارد پس تضمین میشود که مجزا بودن راس ها را داریم.\\
حالا کافیست نشان دهیم
کمترین پوشش مسیری همان پوششی است که بیشترین تعداد یال را دارد.
\\
یک پوشش مسیری با تعداد مسیر مشخص میشود که تعداد یال های هر مسیر
برابر است با:\\
\[\text{تعداد راس های آن مسیر } -1\]
\\
پس تعداد کل یال های این پوشش برابر است با 
\[ \text{تعداد راس ها} - r\]
که $r$ تعداد مسیر هاست.
پس هرچه $r$ کمتر شود تعداد یال ها بیشتر و همچنین
هرچه تعداد یال ها بیشتر یعنی تعداد مسیر ها کمتر بوده.
پس پوشش مسیری با بیشترین یال ممکن همان پوشش مسیری با کمترین تعداد مسیر است.
پس الگوریتم ثابت میشود.
\\\\
تحلیل زمانی:\\\\
تشکیل گراف $S$ از روی $f_e$
های بدست آمده برای یال ها چند جمله ایست و همچنین الگوریتم بیشینه جریان نیز
چند جمله ای حل میشود پس
این الگوریتم چند جمله ای است.


\end{document}