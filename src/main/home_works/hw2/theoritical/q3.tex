\problem{}
ابتدا از روی لیست 
$\text{f}_i$
 ها گرافی جهت دار که راس های آن
کارمندان و یال های آن به صورت جهت دار از هر کارمند به کارمندی است که باید
به آن خبر رسانی کند تشکیل میدهیم.
سپس با الگوریتم $1$ مولفه های قویا همبند را $O(|V|+|E|)$ پیدا میکنیم که در اینجا
تعداد یال ها دقیقا با تعداد راس ها برابر است (زیرا هر کارمندی دقیقا به یک کارمند یال جهت دار دارد.)
پس $O(n)$.\\
ادعا:\\
هر شخصی خبری که میشنود به همه افراد منتقل میشود اگر و تنها اگر تمام افراد در یک مولفه
قویا همبند قرار گیرند.\\
اثبات :\\
اگر تمام افراد در یک مولفه قویا همبند قرار گیرند به وضوح از هر راس (کارمند)
به راس های دیگر مسیری جهت دار هست پس خبر به خوبی پخش میشود.\\
حال میخواهم نشان دهم اگر تمام افراد در یک مولفه قویا همبند نباشند 
فردی وجود دارد که اگر خبر به او برسد همه مطلع نمیشوند.
می دانیم هر dag حداقل یک sink دارد پس
اگر خبر به یکی از افراد در مولفه ای که sink است برسد
خبر فقط در همان مولفه میماند و افراد در مولفه های دیگر از آن خبر
مطلع نمیشوند پس ادعا ثابت میشود.\\
ادعا:\\
کمترین یال مورد نیاز برای همبند کردن گراف dag
برابر است با $max(a,b)$ که a تعداد رأس ها
با درجه ورودی صفر و b تعداد رأس ها با درجه خروجی صفر است.\\
اثبات:\\
اگر از هر رأس به رأس دیگر راه وجود داشته باشد حتما تمامی رأس ها
حداقل یک یال ورودی و خروجی دارند.
پس حداقل به تعداد $max(a,b)$ نیاز داریم.
حالا الگوریتمی ارائه میدهم که با این تعداد 
dag را تبدیل به گراف قویا همبند کند.
توجه کنید که راس هایی که یا درجه ورودی ندارند یا درجه خروجی یا 
سینک هستند و یا سورس .
ابتدا مشخص میکنیم که a بیشتر است یا b 
فرض کنیم که a>b داریم:\\
سورس ها بیشتر از سینک ها در نتیجه هر سینک را به سورس بعدی آن وصل میکنیم
برای ایجاد دور همیلتونی و اگر دو یا چند سورس بعد از سینک مورد نظر بود
سینک مورد نظر را به هر دو وصل میکنیم
به صورت شهودی مانند قطعه های پازل گراف های dag را میچینیم و سینک های قطعه قبلی
را به سورس های قطعه بعدی متصل میکنیم.
با اینکار به وضوح دور همیلتونی تشکیل میشود و تعداد یال های مورد نیاز به تعداد
سورس هاست.
اگر سینک ها بیشتر بود یعنی b>a نیز به صورت مشابه عمل میکنیم و الگوریتمی ارائه دادیم که
با $max(a,b)$ دور همیلتونی ایجاد میکند و اثبات کامل میشود.