\problem{}
ابتدا از روی لیست 
$\text{f}_i$
 ها گرافی جهت دار که راس های آن
کارمندان و یال های آن به صورت جهت دار از هر کارمند به کارمندی است که باید
به آن خبر رسانی کند تشکیل میدهیم.
سپس با الگوریتم $1$ مولفه های قویا همبند را $O(|V|+|E|)$ پیدا میکنیم که در اینجا
تعداد یال ها دقیقا با تعداد راس ها برابر است (زیرا هر کارمندی دقیقا به یک کارمند یال جهت دار دارد.)
پس $O(n)$.\\
ادعا:\\
هر شخصی خبری که میشنود به همه افراد منتقل میشود اگر و تنها اگر تمام افراد در یک مولفه
قویا همبند قرار گیرند.\\
اثبات :\\
اگر تمام افراد در یک مولفه قویا همبند قرار گیرند به وضوح از هر راس (کارمند)
به راس های دیگر مسیری جهت دار هست پس خبر به خوبی پخش میشود.\\
حال میخواهم نشان دهم اگر تمام افراد در یک مولفه قویا همبند نباشند 
فردی وجود دارد که اگر خبر به او برسد همه مطلع نمیشوند.
می دانیم هر dag حداقل یک sink دارد پس
اگر خبر به یکی از افراد در مولفه ای که sink است برسد
خبر فقط در همان مولفه میماند و افراد در مولفه های دیگر از آن خبر
مطلع نمیشوند پس ادعا ثابت میشود.\\
ادعا:\\
اگر در topological sort هر راسی به راس بعد از خود
یال جهت دار داشته باشد و راس اخر را به راس اول به یالی جهت دار وصل کنیم
dag ما با کمترین تعداد ممکن دور دار میشود و همه مولفه ها 
در یک مولفه قرار میگیرند.\\
اثبات:\\
اگر در topological sort بخواهیم بتوانیم از راس i به راس i+1
برویم حتما باید یال جهت دار از i به i+1 وجود داشته باشد
زیرا از هیچ یالی که
