\problem{}
برای اثبات این الگوریتم کافیست دو چیز را ثابت کنیم که به صورت دو ادعا
در زیر ارائه میکنم.\\
% ادعا ۱:\\
% در درخت BFS بدست آمده فاصله بین دو رأس i و j که
% هیچ کدام در سطح آخر نیستند نمیتواند قطر درخت باشد.\\
% اثبات:\\

در ابتدا به این نکته توجه میکنیم که در درخت هر گاه مسیری بین دو رأس داریم
آن مسیر تنها مسیر بین آن دو رأس است.
(زیرا در غیر این صورت دور تشکیل میشود)
درنتیجه مسیری که بین دو رأس i و j  در درخت BFS
وجود دارد ابتدا از i به ریشه میرویم
و اگر j در این مسیر نبود
از j به سمت ریشه میرویم و اولین جایی که به مسیر از i به ریشه برخورد کردیم
میتوانیم به سمت i برویم.
پس هر مسیری یک بخش بالا رونده به سمت ریشه و یک بخش پایین رونده به سمت
رأس مورد نظر دارد.
و نکته دوم قابل توجه این است که قطر درخت هرگز نمیتواند بین i و j ای باشد
که یکی در مسیر دیگری به ریشه است. زیرا فاصله آن رأس تا ریشه بیشتر از فاصله
آن تا رأس دیگر میشود.\\



ادعا ۱:\\
اگر قطر گراف طول D داشته باشد
هر رأس را به عنوان ریشه درخت BFS انتخاب کنیم
حتما یک سر قطر در سطح آخر می افتد.\\
اثبات:\\
رأس v را به عنوان ریشه درخت BFS در نظر میگیریم.
سر اول قطر از آن فاصله $H$ و سر دوم قطر از آن فاصله
$D - H$ دارد.
در این صورت ارتفاع درخت BFS برابر با $max(H,D-H)$ میشود.
زیرا اگر بیشتر شود رأسی به فاصله بیشتر از این دو, از ریشه درخت وجود دارد
که این یعنی آن رأس به عنوان سر دیگر قطر شناخته میشود.\\



% حالا به سراغ اثبات میرویم:\\
% اگر یکی از این دو برگ نباشد به وضوح میتوان از آن به طبقه
% پایین تر رفت و یک واحد به بخش پایین رونده مسیری که قبلا داشتیم یعنی قطر درخت افزود.\\
% حالا اگر هر دو رأس i و j 
% برگ باشند و هیچ کدام در سطح اخر نباشند
% اگر قسمت بالا رونده این مسیر را طی کنیم و برای قسمت پایین رونده
% از جد مشترک آنها به سمت رأس با ارتفاع بیشتر برویم طول مسیر بیشتر و قطر
% درخت زیاد میشود.
% در نهایت اگر جد مشترک i و j نیز به رأس با ارتفاع بیشتر راهی به سمت پایین نداشت
% پس قطعا هیچ کدام از عناصر سطح اخر با i جد مشترک
% پایین تر از جد مشترک i و j ندارند در نتیجه میتوان
% یکی از آنهارا انتخاب کرد و مسیر بین آن و i قطعا بلند تر از مسیر
% بین i و j میشود زیرا هم مسیر بالا رونده که رسیدن به جد مشترک است
% طولانی تر
% و هم مسیر پایین رونده طولانی تر میشود.
% پس ادعا ثابت میشود.\\
ادعا ۲:\\
اگر در درخت BFS یک راس از سطح آخر انتخاب کنیم
و BFS را روی آن اجرا کنیم
مستقل از اینکه آن کدام رأس است عدد یکسانی به عنوان قطر به ما میدهد.\\
اثبات:\\
فرض کنیم $a_1,a_2,..,a_n$ رأس های سطح آخر ما هستند.
اگر $a_i$ راس i را به عنوان سر دیگر قطر خروجی دهد 
اگر $a_j$ را نیز در نظر بگیریم .
یا جد مشترک آن با i همان جد مشترک $a_i$ با i است
که در این صورت همان فاصله را با i دارد.
و یا جد مشترک آن با i متفاوت از جد مشترک $a_i$ و i است که در این
صورت اگر در نظر بگیریم جد مشترک $a_j$ و i پایین تر از جد مشترک
$a_i$ و i است.
اگر i در سطح آخر نباشد آنگاه از جد مشترک i و $a_j$  بجای رفتن 
به i به $a_j$ میرویم.
که این یعنی طول قطر برای $a_i$ بیشتر میشد.
پس اگر جد مشترک متفاوت داشتیم نیز فاصله $a_j$ و $a_i$
با هم را به عنوان خروجی برای $a_j$ در نظر میگیریم.
(زیرا برابر با فاصله $a_i$ و i بود.)
حالا کافیست نشان دهیم که هیج کدام از $a_i$ ها نمیتوانند
عدد بزرگ تری از دیگری به قطر نسبت بدهند. 
این بخش نیز به وضوح معلوم است چون نشان دادیم
هر عددی که از هر کدام از $a_i$ ها بگیریم به سادگی میتوان
به $a_j$ دیگر نیز راسی نسبت داد که همان عدد را بدهد.
پس همگی مساوی و ماکسیمم نداریم.
\\


% و در آخر اثبات مستقل از این بود که کدام رأس به عنوان ریشه درخت BFS انتخاب
% میشود اما برای تکمیل آن نیز ادعا زیر را ثابت میکنم.\\


% ادعا ۳:\\
% اگر قطر گراف طول D داشته باشد
% هر رأس را به عنوان ریشه درخت BFS انتخاب کنیم
% حتما یک سر قطر در سطح آخر می افتد.\\
% اثبات:\\
% رأس v را به عنوان ریشه درخت BFS در نظر میگیریم.
% سر اول قطر از آن فاصله $H$ و سر دوم قطر از آن فاصله
% $D - H$ دارد.
% در این صورت ارتفاع درخت BFS برابر با $max(H,D-H)$ میشود.
% زیرا اگر بیشتر شود رأسی به فاصله بیشتر از این دو, از ریشه درخت وجود دارد
% که این یعنی آن رأس به عنوان سر دیگر قطر شناخته میشود.\\
و اثبات کامل میشود.