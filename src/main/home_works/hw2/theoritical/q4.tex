\problem{}
اشتباهه ممکنه چنتا سینک داشته باشبم.
 طبق الگوریتم ۱ ابتدا مولفه های قویا همبند را برای شهر ها و مسیر هایی که از قبل بین آنهاست
 پیدا میکنیم.
 سپس آنها را topological sort میکنیم
 ممکن است در topological sort ما مولفه های همبندی ای وجود داشته باشند
 که به هیچ وجه به هم راه ندارند و اصلا به هم وابسته نیستند
 این مولفه ها را در دسته های جداگانه ای قرار میدهیم
 سپس برای هر $u_i$ sink آن را در بخشی که هست پیدا میکنیم
 و برای هر $v_i$ sourse آن را.
 حالا اگر $u_i$ و $v_i$ به هم مسیر جهت دار نداشتند 
 یال جهت داری از سینک $u_i$ به سورس $v_i$ وصل میکنیم.
 این کار باعث میشود مسیر های جهت داری بین $u_j$ هایی که از $u_i$ به 
 آنها راه هست به $v_k$ هایی که در مسیر رسیدن به $v_i$ قرار دارند
 ایجاد شود. 
 این کار ممکن است کار مارا برای ساخت مسیر کم کند.
 حالا کافیست درستی الگوریتم را اثبات و زمان آن را تحلیل کنیم.\\
 ادعا:\\
 این الگوریتم به ازای هر i $v_i$ را به $v_j$ مسیر دار میکند.\\
 اثبات:\\
 در هر مرحله با اجرای DFS مشخص میشود که آیا مسیری بین این دو راس هست یا خیر 
 و اگر نباشد به سادگی با وصل کردن سینک $u_i$ به 
 سورس $v_i$ این مسیر ساخته میشود.\\
 ادعا:\\
 این الگوریتم کمترین تعداد مسیر مورد نیاز را به ما میدهد.\\
 اثبات:\\
 برای اثبات این ادعا باید دو چیز را نشان دهیم یکی اینکه اگر در مرحله $i$ ام
 بین دو شهر $u_i$ و $v_i$ مسیری جهت دار نباشد
 حداقل یک مسیر جدید باید ایجاد شود برای اتصال این دو به هم
 و مورد بعد این است که نشان دهیم با هر جایگشتی از $u_i$ ها 
 تعداد مسیر های به دست آمده یکسان است.
 مورد اول بدیهی است پس کافیست نشان دهیم که با هر ترتیبی $u_i$ ها را انتخاب
 و با الگوریتم داده شده مسیری از آن به $v_i$ ایجاد کنیم
 تفاوتی در تعداد مسیر ها ایجاد نمیشود.
 برای اینکار کافیست نشان دهیم در یک جایگشت داده شده با عوض کردن جای دو عنصر
 دلخواه $u_i$ و $u_j$ تفاوتی ایجاد نمیشود.
 چون میدانیم از هر جاگشتی با عوض کردن دو تا دوتا عناصر میتوان به 
 هر جایگشت دلخواه رسید.
 \\
 عناصر را به سه دسته قبل از 
 $u_i$ و بین $u_i$ و $u_j$
 و بعد از $u_j$تقسیم میکنیم
 .\\
 جایگشت اول به صورت 
 $u_1,...,u_i,...,u_j,...,u_m$ و جایگشت دوم به صورت
 $u_1,...,u_j,...,u_i,...,u_m$ است.
 میتوان جایگشت ها را به صورت زیر نگاه کرد
 $u_i,...,u_j,...,u_m$ و
 $u_j,...,u_i,...,u_m$
 زیرا قبل از آن دقیقا تغییرات یکسانی را در گراف ایجاد کرده
 چهار حالت را بررسی میکنم:\\
 ۱: $u_i$ و $u_j$ به ترتیب در جایگشت های اول و دوم
 مسیری اضافه نکنند:
  در این حالت وقتی $u_i$ در جایگشت 
 اول مسیر ایجاد نکرده یعنی تا قبل از آن مرحله به $v_i$ متناظر خود
 وصل شده پس در جایگشت دوم نیز لزومی به اتصال ندارد و همچنین برای
 $u_j$. پس این دو عنصر از جایگشت بی اثر میشوند و هر دو جایگشت یکی است.
 \\
 ۲:$u_i$ در جایگشت اول مسیر ایجاد کند و $u_j$ در جایگشت دوم
 مسیر ایجاد نکند:
در این حالت در جایگشت اول مسیر ایجاد شده توسط $u_i$ باعث بی اثر شدن
عناصری در جایگشت شده که سینک آنها با سینک $u_i$ و سورس آنها با سورس $v_i$
یکی بوده است. درنتیجه 
در جایگشت دوم اگر یکی از این عناصر زودتر از $u_i$ ظاهر شود
$u_i$ و همه آن عناصر را بی اثر کرده
و تعداد مسیر ها ثابت میماند.
برای $u_j$ چون در جایگشت دوم بی اثر است در جایگشت اول نیز (مانند استدلال قسمت قبل)
بی اثر است.
در نتیجه دو حالت دیگر که به صورت زیر هستند نیز با همین استدلال ثابت میشوند.
\\
۳:$u_j$ در جایگشت اول مسیر ایجاد کند و $u_i$ در جایگشت دوم
 مسیر ایجاد نکند.
\\ 
۴:$u_i$ و $u_j$ به ترتیب در جایگشت های اول و دوم
مسیری اضافه کنند.\\
پس ادعا ثابت میشود .
\\ تحلیل زمانی:\\
در هر مرحله برای پیدا کردن مولفه های قویا همبند و topological sort به $O(n)$ زمان نیاز داریم
(که n مجموع راس و یال های گراف است.)\\
و اما در نهایت برای m راس باید این کار را انجام دهیم پس
$O(nm)$ زمان مورد نیاز برای اجرای این الگوریتم میباشد.






% ابتدا بر مولفه های قویا همبند گراف داده شده را پیدا میکنیم.
% سپس dag به دست آمده را به صورت topological sort در می آوریم
% سپس بجای وصل کردن مستقیم یو ۱ به وی ۱ سینک یو یک را به سورس 
% وی یک وصل میکنیم
% اینگونه مسیری جهت دار با یک یال ایجاد شده
% و تمام ui ها را در ان مولفه ای که به صورت توپولوژیکال سورت هستند
% به هم وصل کرده و سپس برای یو دو اگر سینک آن به سورس وی دو وصل نبود 
% وصل میکنیم و این ترتیب بهترین ترتیب
% است. 
% حالا ثابت کن این ترتیب بهترینه.