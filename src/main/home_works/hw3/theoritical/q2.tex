\problem{}
گراف جهت دار بدون دور یا همان DAG به ما یک
توپولوژیکال سورت میدهد
که به این ترتیب عمل میکنیم:\\
\begin{enumerate}
    \item 
    اولین راس در توپولوژیکال سورت
    را برداشته و اگر این راس به راسی یال داشت
    آن راس را برمیداریم و راس های مجاور آن راس را حذف
    سپس در گراف باقی مانده اگر باز هم
    راس اول با راس دیگری مجاور بود آن راس را نیز برداشته و دوباره
    راس های مجاور آن را حذف میکنیم.
    (زیرا اگر این کار را انجام ندهیم مسیر به طول دو داریم)
    \item
    در بین راس های حذف نشده راس بعدی در توپولوژیکال سورت
    را انتخاب و دوباره مرحله قبل را روی آن اجرا میکنیم.
    \item
    پس از اینکه همه راس ها یا انتخاب شدند یا حذف الگوریتم را متوقف میکنیم.
\end{enumerate}

اثبات درستی:\\\\
ادعا ۱:
الگوریتم مسیر جهت دار به طول دو ندارد.\\\\
اثبات:\\
یکی از راس های انتخاب شده را در نظر میگیریم این راس اگر به راس دیگری یال داشته باشد
آن راس دیگر به هیچ راسی یال ندارد.(زیرا اینگونه راس ها را حذف کردیم)
و درجه ورودی این راس نیز صفر است زیرا اگر درجه ورودی داشت پس از اینکه این راس انتخاب میشد
در فرایند الگوریتم راسی که به آن یال دارد حذف میشد . به همین ترتیب اگر 
راس انتخابی ما درجه خروجی نداشته باشد.
و درجه ورودی داشته باشد راسی که از آن به این راس درجه ورودی هست
درجه ورودی ندارد.
پس مسیر به طول دو نداریم.\\\\
ادعا ۲:
گراف خروجی حداقل $\frac{3n}{7}$ راس دارد.\\\\
اثبات:\\
در روند الگوریتم راسی که داریم روی آن فرایند
حذف یا انتخاب را انجام میدهیم در نظر بگیرید.\\
اگر درجه خروجی این راس ۰ باشد یک راس انتخاب شده و هیچ راسی به ازای آن حذف نمیشود.\\
اگر درجه خروجی این راس ۱ باشد دو راس انتخاب شده و حداکثر دو راس حذق میشود.\\
اگر درجه خروجی آن ۲ باشد دو راس انتخاب شده و حداکثر چهار راس حذف میشود.\\
پس در بدترین حالت در هر مرحله اگر در نظر بگیریم ۳ راس انتخاب شده و ۴ راس حذف
نسبت انتخاب شده ها به کل راس ها مینیمم $\frac{3}{7}$
می باشد که ادعا ثابت میشود.