\problem{}
برای حل این سوال یک گراف جهت دار کمکی میسازیم به این صورت که
 ابتدا روی هر راس دایکسترا میزنیم 
 و فاصله آن راس از راس های دیگر به دست می آید و حالا به راس هایی که فاصله آنها کمتر مساوی
 طولی که تاکسی میتواند طی کند است میتوان با آن تاکسی رفت.
 پس در گراف جدید راسی که الان روی آن هستیم را در نظر میگیریم
 و از آن راس به راس هایی که میتوان با آن تاکسی رفت یک یال جهت دار به وزن هزینه تاکسی رسم میکنیم
 به ازای همه راس ها اینکار را انجام میدهیم.
 حالا در گراف جدید 
 هزینه های مختلفی که از هر راس میتوان به راس دیگر رفت به حالت های مختلف را داریم
 کافیست روی گراف جدید دایکسترا از s بزنیم و مینیمم
 فاصله آن تا t را محاسبه کنیم.
 که همان مینیمم هزینه است زیرا وزن های گراف جدید بر اساس هزینه هاست.
\\
تحلیل زمانی الگوریتم:\\
روی هر راس یک دایکسترا اجرا کردیم که اردر هر دایکسترا $O(m\log{n})$
است. پس حالا که $n$ بار آن را اجرا کردیم داریم:\\
$O(nm\log{n})$.\\
حالا در گراف جدید اگر حتی گراف کامل باشد 
$O(n^2)$ یال داریم که این یعنی اجرا دایکسترا روی آن
$O(n^2\log{n})$
است.\\
در نتیجه زمان اجرای کل الگوریتم برابر است با $O((n^2+mn)\log{n})$.\\\\
اثبات درستی:\\
درستی این الگوریتم به وضوح اثبات میشود
به این صورت که 
به ازای هر تاکسی تمام مسیر هایی که میتوان با آن تاکسی طی کرد به صورت 
جهت دار مشخص شده و هزینه آن نیز الحاق شده
و درنهایت تمام مسیر های قابل طی با هزینه های آنها در نظر گرفته شده
و با اجرای دایکسترا روی آن گراف روی راس s کوتاه ترین
مسیر از لحاظ هزینه ای مشخص میشود.