\problem{}
این سوال را خیلی سال پیش یک انسان به اسم جانسون حل کرده
و من صرفا راه حل هوشمندانه او را بازگو میکنم:)\\
ابتدا یک راس دلخواه انتخاب کرده و الگوریتم بلمن فورد را از آن راس اجرا میکنیم.
این راس را s مینامم.
حالا به هر راس i یک عدد $p_i$ نسبت میدهیم که
طول کوتاه ترین مسیر از s به آن راس است.
گراف جدیدی معرفی میکنیم که همه یال های آن وزن مثبت دارند
به صورتی که $l^{\prime}_{uv} = l_{uv}+p_u - p_v$ می باشد
حالا در گراف جدید n بار الگوریتم دایکسترا را اجرا میکنیم
تا اینکه کوتاه ترین طول ها را برای تمام جفت راس ها محاسبه کنیم.
و در نهایت طول بدست آمده برای هر مسیر را به مقدار واقعی آن تبدیل میکنیم.\\
اثبات درستی:\\
ادعا یک:\\
طول تمام یال ها در گراف جدید مثبت است.\\
اثبات:\\
فرض کنیم یال (u,v) را داریم.
میدانیم $p_v\leq p_u+l_{uv}$ است در نتیجه 
داریم:\\
\[ 0\leq l_{uv}+p_u- p_v\]
ادعا دو:\\
طول کوتاه ترین مسیر از $a_1$ به $a_r$ تحت این تغییر وزن تغییر نمیکند:\\
اثبات:\\
فرض کنیم مسیر $a_1,a_2,...,a_r$ را داریم
طول جدید این مسیر برابر میشود با:\\
\[
    \sum_{i = 1}^{r-1}{l_{a_i a_{i+1}} + p_{a_i}-p_{a_{i+1}}}
\]
که برابر است با طول قبلی این مسیر به اضافه $p_{a_1}-p_{a_r}$
که این یعنی مستقل از مسیر طول تمامی مسیر ها به یک اندازه افزایش یافته
و درنتیجه کوتاه ترین مسیر همان قبلی میماند.
 
پس الگوریتم درست کار میکند و کافیست پس از n بار اجرای دایکسترا روی تمام راس ها
از طول کوتاه ترین مسیر به دست آمده بین راس های u و v عبارت 
$p_{u}-p_{v}$ را کم کنیم.
\\تحلیل زمانی:
زمان اجرای بلمن فورد هست $O(mn)$ و زمان اجرای n بار دایکسترا هست
$O(nm\log{n})$ و درنهایت محاسبه طول واقعی از طول های بدست آمده برابر است با $O(n^2)$ که زمان الگوریتم میشود
$O(nm\log{n})$.