\problem{}
ابتدا یک گراف کمکی میسازم که راس های آن راس های علامت دار گراف داده شده هستند.
سپس
از یک راس علامت دار دلخواه شروع کرده و dfs را اجرا میکنم تا به راس
علامت دار دیگری برسم
سپس بین این دو راس علامت دار یک یال به وزن فاصله این دو راس اضافه میکنم و همچنین یال های پیمایش شده را
در dfs حذف میکنم
حالا یک راس علامت دار دیگر انتخاب کرده
و در صورتی که مسیری از آن وجود داشت آن را دنبال کرده تا به راس 
علامت دار دیگری برسم یا در dfs به خود آن راس برگردم و تمام یال های مسیر
را حذف کنم.
در نهایت این کار را انقدر ادامه میدهم تا گراف به گراف بدون یال تبدیل شود.
حالا در گراف به دست آمده وزن یال ها را در هم ضرب و به عنوان جواب سوال خروجی میدهم.
\\\\
اثبات درستی:\\\\
چونکه گراف داده شده یک درخت است بین هر دو راس علامت دار دقیقا یک مسیر
وجود دارد حالا من در گراف کمکی به صورتی دارم راس های علامت دار را به عنوان مولفه
همبندی در نظر گرفته و بین دو راس علامت داری که یک مسیر مستقیم وجود دارد(مسیری که در آن راس علامت دار دیگری نیست)
یک یال به وزن طول آن مسیر در نظر میگیرم.
و درخت بودن گراف اولیه باعث میشود گراف
علامت دار جدید نیز درخت باشد.
حالا دو راس علامت دار مستقیم را در نظر بگیرید که بین آنها در گراف کمکی یال هست
مثلا وزن این یال r است.
این یعنی r یال در گراف اصلی بین این دو راس بوده
حالا ما یکی از این r یال را انتخاب و حذف کرده
و راس های که متصل به راس علامت دار اول میمانند
در مولفه همبندی آن قرار میگیرند و بقیه در مولفه همبندی
راس دوم قرار میگیرند.
با توجه به این که راس های این مسیر باید حتما در یکی از این دو ملفه همبندی قرار میگرفتند
و از یک جایی ما باید برش میزدیم و دقیقا r حالت داریم
پس با ضرب تمام این وزن ها در هم تمام حالت ها به پوشش داده میشود.
(دقت کنید که گراف کمکی صرفا برای درک بهتر مولفه های همبندی بوده و کمک در اثبات درستی وگرنه میتوانستیم
مستقیما اعداد بدست آمده را در هم ضرب کنیم).
\\\\
تحلیل زمانی:\\\\
الگوریتم dfs تعمیم یافته ای که اجرا میکنیم در مجموع
به ازای هر یال $O(1)$ عملیات انجام میدهد.
به صورت دقیق تر یال های هر مسیر را نگه میدارد تا به راس علامت دار اولیه برسد
و یا به راس علامت دار دیگری سپس آن یال های نگه داشته شده را از گراف حذف میکند.
که این برابر $O(\text{طول آن مسیر})$
زمان میبرد.
و در نهایت مجموع همه مسیر ها میشود $O(m+n)$
که چون گراف داده شده درخت است برابر است با 
$O(n)$ و در نهایت گراف کمکی در بدترین حالت n-1 یال دارد
که پیمایش روی آنها نیز $O(n)$ است.
پس زمان اجرایی کل الگوریتم برابر است با $O(n)$.